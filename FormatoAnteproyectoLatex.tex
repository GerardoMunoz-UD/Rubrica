\documentclass{article}
\usepackage[spanish]{babel}
\usepackage{color,graphicx} % Required for inserting images

\title{Universidad Distrital Francisco José de Caldas \\ Facultad de Ingeniería}
\author{Formato de Propuesta de Modalidad de Grado}
%\date{June 2023}

\begin{document}

\maketitle

\section*{\sc Identificación Básica del Proyecto}
\subsection*{Título del Anteproyecto}
\noindent Escriba aquí el título del Proyecto.

\subsection*{Modalidad}
\noindent Seleccione una y solo una de las opciones siguientes:  

\medskip \noindent Pasantía \fbox{\color{white}X} \quad Monografía \fbox{\color{white}X} \quad Emprendimiento \fbox{\color{white}X} \quad  Artículo Académico \fbox{\color{white}X} \quad Investigación, Investigación -- Creación, Innovación \fbox{\color{white}X}

\subsection*{Estructura de Investigación}
\noindent Si la modalidad corresponde a Investigación, seleccione el escenario donde desarrolla su trabajo y el nombre de la organización:

\medskip \noindent Semillero de Investigación \fbox{\color{white}X} \quad Grupo de Investigación \fbox{\color{white}X} \quad Instituto de Investigación \fbox{\color{white}X} \quad Nombre: \fbox{\color{white}X\hspace{20em} X}

\subsection*{Tipo de Artículo de Investigación}
\noindent Si corresponde a la modalidad escogida, seleccione una de las opciones siguientes:

\medskip \noindent Científico \fbox{\color{white}X} \quad Tecnológico \fbox{\color{white}X} \quad De Reflexión \fbox{\color{white}X} \quad De Revisión \fbox{\color{white}X} \quad Reporte de Caso \fbox{\color{white}X} \quad Revisión de Tema Propuesto \fbox{\color{white}X}

\subsection*{Temática Asociada al Perfil Profesional}
\noindent \fbox{\color{white}X\hspace{33em}}

\section*{\sc Identificación de Estudiantes y Docentes}
\subsection*{Estudiante 1}
\noindent Nombre: \fbox{\color{white}X\hspace{28em}} \\
Código: \fbox{\color{white}X\hspace{6em}} \quad Correo Institucional: \fbox{\color{white}X\hspace{11em}}
\subsection*{Estudiante 2}
\noindent Nombre: \fbox{\color{white}X\hspace{28em}} \\
Código: \fbox{\color{white}X\hspace{6em}} \quad Correo Institucional: \fbox{\color{white}X\hspace{11em}}
\subsection*{Docente que avala la propuesta}
\noindent Nombre: \fbox{\color{white}X\hspace{28em}} \\
Correo Electrónico Institucional: \fbox{\color{white}X\hspace{11em}}
\subsection*{Profesional responsable del acompañamiento {\scriptsize(solo Pasantía)}}
\noindent Nombre: \fbox{\color{white}X\hspace{28em}} \\
Correo Electrónico: \fbox{\color{white}X\hspace{11em}}
\subsection*{Entidad Receptora del Pasante {\scriptsize{(solo Pasantía)}}}
\noindent Nombre: \fbox{\color{white}X\hspace{28em}}

\section*{\sc Propuesta}
\subsection*{Formulación del Problema}
\noindent Aquí se describe formalmente el problema que va a tratar el proyecto. Posteriormente, se describe, también formalmente, cómo se espera solucionar el problema. Formalmente quiere decir que la descripción del problema y de la propuesta de solución deben basarse en modelos abstractos propios del escenario del problema. Cada área de la ingeniería posee los elementos abstractos con los cuales modela los diferentes objetos que estudia.

En general, los modelos abstractos típicos de ingeniería son matemáticos puesto que la ingeniería se basa en modelos físicos. Al definir modelos, varios elementos del proyecto surgen naturalmente como, por ejemplo, la definición de las variables, los procesos, sus descripciones y notación, entre otros. Estos modelos formales son los que constituyen el marco teórico del proyecto puesto que recurren a modelos teóricos ya estudiados profundamente (esta es una de las razones por las cuales esta sección debería contener un buen número de referencias bibliográficas). % Por ejemplo, si el problema fuera la construcción de sistemas lógicos basados en estados lógicos, el álgebra de Boole es la matemática apropiada para estos objetos. 
Más adelante, en la sección {\bf Marco Referencial}, 

Cuando la modalidad corresponde a un proyecto de investigación, también se deben incluir las preguntas de investigación y una hipótesis, que es la que se va a probar. Puesto que en el método científico se requiere validar la hipótesis, es necesario también proponer los procesos de verificación de la hipótesis.

\subsection*{Objetivo General}
\noindent Aquí se escribe el objetivo general del proyecto. Consiste en una oración que comienza con un verbo en infinitivo. Esto es porque en la escritura de los objetivos se omite la primera parte de la oración que podría decir \guillemotleft El propósito del proyecto es... \guillemotright. Debe tenerse en cuenta que el objetivo general determina lo que es el proyecto y debe describir objetivamente lo que se espera después de desarrollarlo. De hecho, lo que se obtiene será lo que se utilice para evaluar el resultado del proyecto. % Por ejemplo, si el propósito del proyecto es construir un circuito que hace alguna labor de control de regulación sobre una máquina, el objetivo debería decir: Construir un circuito que realice el control de regulación de la máquina tal...

\subsection*{Objetivos específicos}
\noindent Aquí se listan los objetivos específicos que, en general, corresponden a diferentes etapas del desarrollo del proyecto. Ciertamente, la conjunción de todos los objetivos específicos deben llevar a lograr el objetivo general. Los objetivos específicos también se escriben comenzando con verbos en infinitivo y deben corresponder con lo que se espera obtener de cada uno de ellos. Todo trabajo de grado, de los planteados aquí, debe concluir con la elaboración de un reporte o informe final con todas las secciones que se establecen para estos reportes. % En general, un reporte final contiene una introducción, unos antencedentes, uno modelado del problema que se estudia o resuelve, una propuesta de solución consistente con el modelo del problema, la descripción de la implementación de la solución, las pruebas y resultados de las pruebas realizadas, las interpretaciones de los resultados, una discusión sobre estos resultados, unas conclusiones. Todo esto se construye en un documento que corresponde al reporte del trabajo realizado para cumplir con el proyecto.

\subsection*{Justificación}
\noindent En esta sección se escribe por qué es importante hacer el proyecto. Puede haber diferentes tipos de justificaciones. Las justificaciones pueden ser académicas, sociales, económicas, técnicas, etc. Aunque todos los tipos de justificaciones inciden, de alguna manera, en cualquier proyecto, siempre hay una que es determinante dentro del proyecto. Seleccione uno o dos tipos nada más. % Por ejemplo, un proyecto podría tener un propósito pedagógico (justificación académica); sin embargo, aunque podrían obtenerse beneficios económicos, no sería realmente el propósito obtener prototipos más baratos (justificación econónica). Esto hace que la meta del proyecto sea clara y las decisiones que se tomen están acordes con esta.

\subsection*{Descripción de la idea de emprendimiento \\ {\scriptsize(Solo aplica para la modalidad de emprendimiento)}}
\noindent Aquí se escribe por qué es importante crear la empresa que desarrolla el producto o los productos... Adicionalmente, se escribe cómo se construiría esta empresa y los productos que elaboraría. 

\subsection*{Antecedentes}
\noindent En esta sección se describen los trabajos que se han desarrollado hasta el momento y que tienen que ver con el proyecto. Esta sección debe contener bastantes referencias bibliográficas donde se demuestre un revisión documental exhaustiva del tema. Es conveniente que este estudio se escriba de manera cronológica, es decir, que los trabajos o reportes más antiguos vayan primero. Esto permitiría mostrar una \guillemotleft evolución\guillemotright\ en el desarrollo del problema que se quiere resolver con este trabajo.

\subsection*{Marco Referencial}
\noindent El marco referencial se refiere a los aspectos teóricos que se relacionan con la propuesta. Aquí se especifican esos aspectos teóricos que están involucrados. Sin embargo, se deben relacionar con lo que se va a desarrollar en el proyecto. No es solamente escribir teoría. Es teoría relacionada con lo que se va a desarrollar dentro del proyecto. Esta sección debe demostrar que ya se ha hecho un estudio previo de la propuesta con los elementos teóricos que se utilizarán en el proyecto. 

\subsection*{Plan de Trabajo \\ {\scriptsize (Individual para pasantía)}}
\noindent Aquí se listan las diferentes actividades que el pasante va a realizar durante la pasantía.

\subsection*{Cronograma}
\noindent En esta sección se describen y listan las diferentes actividades a realizar durante el desarrollo del proyecto (de acuerdo con la modalidad, por supuesto). A cada una de estas actividades se le debe estimar un tiempo de ejecución y una ubicación temporal dentro del desarrollo del proyecto.

Las actividades listadas en el cronograma deben reflejar el cumplimiento de los objetivos tanto específicos como general. Además, debe permitir evaluar el cumplimiento de los términos especificados en la lista tanto para cada una de las actividades como para el proyecto general. En caso de requerir una prórroga, el análisis del cronograma puede ser considerado para aceptar o rechazar el requerimiento.

La asignación de tiempo a cada una de las actividades debe hacerse cuidadosamente. Generalmente, las estimaciones en el tiempo de ejecución son mayores que las que uno se imagina. De hecho, siempre será mejor tener suficiente tiempo incluso para las posibles eventualidades que puedan atrasar su ejecución. Además, si un proyecto termina antes, da la posibilidad de revisarlo.

\subsection*{Presupuesto}
\noindent 


\end{document}
