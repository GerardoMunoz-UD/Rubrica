\documentclass{article}
\usepackage{graphicx} % Required for inserting images
\usepackage[spanish]{babel}

\title{Habilidades resultados de aprendizaje}
%\author{gmunoz.udistrital.edu.co }
%\date{July 2023}

\begin{document}

%\maketitle

%\section{Introduction}

Este  documento es un borrador de trabajo

\section{Propuesta de tabla de habilidades ¿competencias?}

Se propone una tabla de habilidades y se relaciona con las dos tablas de resultados de aprendizaje.

\begin{table}[h!]
\begin{tabular}{|l|l|l|ll}
\hline 
Habilidades                    & ra    & RA            \\
\hline 
\hline 
Teóricas                       & 1,7   & 1,2      \\
\hline 
legales y   normativa          &       &               \\
\hline 
culturales   y ecológicas      & 3,8   & 9,4,11        \\
\hline 
comerciales   y económicas     & 3,5,7 & 2,4,6,8,4,11  \\
\hline 
para   producción de productos & 2     & 3             \\
\hline 
comunicativas                  &       &               \\
\hline 
éticas                         & 9     & 10            \\
\hline 
colaborativas                  & 4     & 10            \\
\hline 
experimentales                 &       & 5             \\
\hline 
de   autoaprendizaje           &       & 12            \\
\hline 
creativas                      & 6     & 7             \\
\hline 
de   gestión                   &       & 5             \\
\hline 
operativas                     &       &               \\
\hline 
artísticas                     &       &              \\
\hline 
\end{tabular}
\end{table}

\section{Primera propuesta de  resultados de aprendizaje, llamada en minúsculas \textbf{ra}}
\begin{itemize}
 \item ra1	Conoce los principios que fundamentan la ingeniería electrónica y su campo de acción en la sociedad
 \item ra2	Comprende el diseño de circuitos electrónicos fundamentales con el fin de contextualizarlos dentro una solución factible y realizable
 \item ra3	Aplica los fundamentos y conocimientos de la electrónica para resolver problemas que existen en los campos de acción de la ingeniería
 \item ra4	Analiza problemas de ingeniería utilizando modelos, mediante el trabajo en equipo y el aprendizaje basado en proyectos
 \item ra5	Sintetiza los conocimientos aprendidos durante su formación para resolver problemas relacionados con el sector productivo
 \item ra6	Implementa sistemas electrónicos mediante hardware y software de manera original e innovadora.
 \item ra7	Evalúa a partir del pensamiento crítico la síntesis de proyectos fundamentándose en aspectos técnicos, administrativos y socio económicos
 \item ra8	Presenta propuestas de solución en el campo de la ingeniería electrónica de manera asertiva  de acuerdo con las necesidades particulares que se presentan en los territorios y campos de acción, propiciando la sostenibilidad ambiental y social.
 \item ra9	Desarrolla eficientemente procesos propios de la ingeniería electrónica de forma individual, en equipo y/o en ambientes multidisciplinarios basados en los principios éticos de la ingeniería.

\end{itemize}



\section{Segunda propuesta de  resultados de aprendizaje, llamada en mayúsculas \textbf{RA}}

\begin{itemize}
  \item RA1   Muestra el entendimiento de las matemáticas, la física y la computación como elementos   básicos para el aprendizaje de los fundamentos de la Ingeniería Electrónica.                                                                                   
  \item RA2.   Conoce los principios que fundamentan la ingeniería electrónica y su campo de   acción en la sociedad.                                                                                                                                                 
  \item RA3.   Comprende el diseño de circuitos electrónicos fundamentales con el fin de   contextualizarlos dentro una solución factible y realizable.                                                                                                               
  \item RA4.   Aplica los fundamentos y conocimientos de la electrónica para resolver   problemas que existen en los campos de acción de la ingeniería.                                                                                                               
  \item RA5.   Analiza problemas de ingeniería utilizando modelos, mediante el trabajo en   equipo y el aprendizaje basado en proyectos.                                                                                                                              
  \item RA6.   Sintetiza los conocimientos aprendidos durante su formación para resolver   problemas relacionados con el sector productivo.                                                                                                                           
 \item  RA7.   Implementa sistemas electrónicos mediante hardware y software de manera   original e innovadora.                                                                                                                                                       
 \item  RA8.   Evalúa a partir del pensamiento crítico la síntesis de proyectos   fundamentándose en aspectos técnicos, administrativos y socio económicos.                                                                                                           
  \item RA9.   Presenta propuestas de solución en el campo de la ingeniería electrónica de   manera asertiva de acuerdo con las necesidades particulares que se presentan   en los territorios y campos de acción, propiciando la sostenibilidad   ambiental y social.
  \item RA10. Desarrolla   eficientemente procesos propios de la ingeniería electrónica de forma   individual, en equipo y/o en ambientes multidisciplinarios basados en los   principios éticos de la ingeniería.                                                    
 \item  RA11.   Identifica cómo los conocimientos sociales y económicos impactan en el   desarrollo desolución de problemas de Ingeniería Electrónica.                                                                                                                
  \item RA12.   Conoce la importancia del auto-aprendizaje continuo como elemento que le   permite extender su conocimiento y mantenerse actualizado, al igual que   identificar y utilizar diferentes                                                             
\end{itemize}

\end{document}
